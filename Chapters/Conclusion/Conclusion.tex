% Chapter 1

\chapter{Conclusions} % Main chapter title

\label{Conc} % For referencing the chapter elsewhere, use \ref{Chapter1} 

%----------------------------------------------------------------------------------------

% Define some commands to keep the formatting separated from the content 
\newcommand{\keyword}[1]{\textbf{#1}}
\newcommand{\tabhead}[1]{\textbf{#1}}
\newcommand{\code}[1]{\texttt{#1}}
\newcommand{\file}[1]{\texttt{\bfseries#1}}
\newcommand{\option}[1]{\texttt{\itshape#1}}

%----------------------------------------------------------------------------------------

\section{Review of Project Goals}
%This `baseline solution' finds the first $k$ eigenstates of the Hamiltonian associated with an elecron interacting with a Hydrogenic atom at an instant, with the Coulomb Potential being modelled instead by a Soft-Core potential - which at all but very small distances is an accurate approximation of the Coulomb Potential. Due to the Coulomb Potential tending towards $\infty$ as $r\rightarrow 0$ it is difficult to model it numerically, but with a Soft-Core Potential approximation, we avoid the issues at small $r$s yet obtain accurate results for everywhere other than at the center.

%Is this an appropriate place for the above?

%The Radial TISE is a simplification of the TISE for the case where the system is spherically symmetric, which is the case for a Hydrogenic atom - for both a Coulomb Potential or a Soft-Core Potential.



\section{Overview of Findings}
\subsection{TISE}
adslkjn 
\subsection{TDSE}
asdf 
\subsection{RMT}
kdadfkj

\section{Potential Further Research Directions}

\subsection{Complete FD RMT Outer Region Propagator}
kjna kma

\subsection{Add Functionality To FD RMT Implementation}
fs hsdf 
%\subsection{Shared \& Distributed Memory}
%dfgaljsdf lkjdhf adslkjn asdja alksjnda lknas lakn KJN Kdsf lk

\subsection{Investigate `Non-Finite' Difference Model}
asdbb

%----------------------------------------------------------------------------------------
